To solve this problem we plot the graph of \(f(x)\) for all the four cases using GNU Octave. \\~\\

\section*{GRAPHS}
%\vspace{2cm}
%{\huge\bfseries {\color{BrickRed} ---------------------} \(f(x) = \frac{2x^2}{a}\)}\\
%{\huge\bfseries {\color{Cerulean} ---------------------} \(f(x) = a\)}\\
%{\huge\bfseries {\color{NavyBlue} ---------------------} \(f(x) = \frac{2b^2 - 4b}{x^3}\)}
%\par
%\includegraphics[scale = 1]{P_19_1.eps}\\
\(f(x)\) is a continuous function when \(a = \sqrt{2}\) and \(b= 1 - \sqrt{3}\). \par
%\includegraphics[scale = 1]{P_19_2.eps}\\
\(f(x)\) is a discontinuous function when \(a = - \sqrt{2}\) and \(b= 1 + \sqrt{3}\).\\ It is discountinuous at \(x = \sqrt{2}\). \par
%\includegraphics[scale = 1]{P_19_3.eps}\\
\(f(x)\) is a discontinuous function when \(a = \sqrt{2}\) and \(b= -1 + \sqrt{3}\).\\ It is discountinuous at \(x = \sqrt{2}\). \par
%\includegraphics[scale = 1]{P_19_4.eps}\\
\(f(x)\) is a discontinuous function when \(a = - \sqrt{2}\) and \(b= 1 - \sqrt{3}\).\\ It is discountinuous at \(x = \sqrt{2}\). \par

\section*{GNU Octave's CODE}
\subsection{For 1)}
\begin{lstlisting}[frame = single]
\end{lstlisting}
\subsection{For 2)}
\begin{lstlisting}[frame = single]
clear;
close;
clc;

a = -sqrt(2); b = 1 + sqrt(3);
x1 = linspace(0, 1, 1000);
x2 = linspace(1, sqrt(2), 1000);
x3 = linspace(sqrt(2), 5, 100000);

y1 = 2.*(x1.^2)./a;
y2 = a;
y3 = (2*(b^2) - 4*b)./(x3.^3);

plot(x1, y1, "2", x2, y2, "5", x3, y3, "3");
grid minor;
title("Problem 19\n 
     2.)  a = -sqrt(2); b = 1 + sqrt(3)\n
     f(x) is Discontinuous");
xlabel("x");
ylabel("f(x)");
\end{lstlisting}
\subsection{For 3)}
\begin{lstlisting}[frame = single]
clear;
close;
clc;

a = sqrt(2); b = -1 + sqrt(3);
x1 = linspace(0, 1, 1000);
x2 = linspace(1, sqrt(2), 1000);
x3 = linspace(sqrt(2), 5, 100000);

y1 = 2.*(x1.^2)./a;
y2 = a;
y3 = (2*(b^2) - 4*b)./(x3.^3);

plot(x1, y1, "2", x2, y2, "5", x3, y3, "3");
grid minor;
title("Problem 19\n 
     3.)  a = sqrt(2); b = -1 + sqrt(3)\n
     f(x) is Discontinuous");
xlabel("x");
ylabel("f(x)");
\end{lstlisting}
\subsection{For 4)}
\begin{lstlisting}[frame = single]
clear;
close;
clc;

a = -sqrt(2); b = 1 - sqrt(3);
x1 = linspace(0, 1, 1000);
x2 = linspace(1, sqrt(2), 1000);
x3 = linspace(sqrt(2), 5, 100000);

y1 = 2.*(x1.^2)./a;
y2 = a;
y3 = (2*(b^2) - 4*b)./(x3.^3);

plot(x1, y1, "2", x2, y2, "5", x3, y3, "3");
grid minor;
title("Problem 19\n 
     4.)  a = -sqrt(2); b = 1 - sqrt(3)\n
     f(x) is Discontinuous");
xlabel("x");
ylabel("f(x)");
\end{lstlisting}
%\end{flushleft}
%\end{document}
