From the given information, 
\begin{align}
\label{one_1}
f_{1}(x)&=f_{0}(f_{0}(x))=\frac{1}{1-\frac{1}{1-x}}=\frac{1-x}{-x}, \\
\label{one_2}
f_{2}(x)&=f_{0}(f_{1}(x))=\frac{1}{1-\frac{1-x}{-x}}=x, \\
\label{one_3}
f_{3}(x)&=f_{0}(f_{2}(x))=\frac{1}{1-x}=f_{0}(x), \\
\label{one_4}
f_{4}(x)&=f_{0}(f_{3}(x))=\frac{1}{1-\frac{1}{1-x}}=\frac{1-x}{-x}=f_{1}(x)
\end{align}
The function repeats in a similar manner for other values of $n$ as well.
From \eqref{one_1},\eqref{one_2}, \eqref{one_3} and \eqref{one_4},
\begin{align}
  f_{100}(3)&=f_{1}(3)=\frac{1-3}{-3}=\frac{2}{3}\\
  f_{1}\left(\frac{2}{3}\right)&=\frac{1-\frac{2}{3}}{-\frac{2}{3}}=\frac{-1}{2}\\
  f_{2}\left(\frac{3}{2}\right)&=\frac{3}{2}
\end{align}
resulting in
%
\begin{equation}
f_{100}(3) + f_{1}\left(\frac{2}{3}\right) + f_{2}\left(\frac{3}{2}\right)=\frac{2}{3} + \frac{-1}{2} + \frac{3}{2}=\frac{5}{3}	
\end{equation}
%
