\begin{flushleft}
Finding the first-derivative of the given equation gives 
\[y' =-2 sin(2x) cos(2x)\]
\\
Plotting region of first-derivative i.e \(y' =-2 sin(2x) cos(2x)\)\\
%\includegraphics[scale=0.8]{final.jpg}\\
Observing the preceding graph , the interval in which graph \[y' >=0\]
gives \[[0.79,1.57]  and  [2.36,3.14]\] \\
Finally plotting the given equation \(f(x) = sin^4(x) + cos^4(x)\)\\
%\includegraphics[scale=0.5]{one.jpg}
%\\
and then  shadding the region where graph increases
%\includegraphics[scale=0.8]{graph.jpg}

\end{flushleft}


\vspace{78mm}

\section{Code }
\lstset{language=Octave}

\begin{lstlisting}[frame=single]
\end{lstlisting}


\begin{lstlisting}[frame=single]
clear;
close;
clc;
x=linspace(0,pi,10000);
y=-2*sin(2*x).*cos(2*x);


plot (x,y);
hold on;
y1=y;
y1(y1<0)=0;
area(x,y1);
grid minor;
xlabel('0<x<pi'); 
ylabel('Y-axis');
title('y'' = -2sin(2x)cos(2x)');
legend('Region where Derivative >=0');
\end{lstlisting}

%\end{document}
